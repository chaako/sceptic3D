\newif \iftth
\iftth 
\else
\parindent= 0 pt
\parskip = 2 pt
\pretolerance=9000
\overfullrule 0 pt
\rightskip 0pt plus0pt minus1pt
\fi

\magnification \magstep2

\centerline{\bf Summary of ACCIS plotting system routines}

\centerline{I.H.Hutchinson}
\bigskip

\beginsection{Conventions}

For purposes of this summary, the following implicitly indicate types:
i: integer, c: character, l: logical, all others: real.
Terminating letter indicates: v: vector (1-d array), a: 2-d array.

\beginsection{Automatic plotting, single call plots.}

AUTOPLOT(xv,yv,ilength) Line plot yv versus xv; linear axes.

AUTOMARK(xv,yv,ilength,isym) Plot using symbol isym.

LAUTOPLOT(xv,yv,ilength,lnx,lny) Line plot with logarithmic axis in x or y.

LAUTOMARK(xv,yv,ilength,lnx,lny,isym) Symbol plot with logarithmic axis(es).
 
YAUTOPLOT(yv,ilength) Line plot yv versus index; linear axes.


\beginsection{Setup}

PLTINIT(xmin,xmax,ymin,ymax) Switch to graphics, set world scaling.

FITINIT(xmin,xmax,ymin,ymax) same but fitting nice ranges.

SCALEWN(xmin,xmax,ymin,ymax,lnx,lny) Set world scaling, possibly log.

FITSCALE(xmin,xmax,ymin,ymax,lnx,lny) same but fitting nice ranges.

PLTINASPECT(xmin,xmax,ymin,ymax) Initiate graphics. Resize axis region
to retain aspect ratio of plot. Set world scaling.

AXREGION(xin,xan,yin,yan) Set the axis region in normal units.
Default: .31,.91,.1,.7

AXPTSET(xfrac,yfrac) Set the axis point (i.e. intersection) at specified
fraction of the plot region. (Default 0.,0.)

MULTIFRAME(irows,icolumns,ispace) For subsequent plots (using pltinit),
plot irows x icolumns plots per page, with label space determined by
ispace (bit0 implies space x, bit1 y). Call with irows=0, single frame.

PFSET(inum) Set file plotting for hardcopy. inum=(0:no, 1:hp, 2:ps, 3:eps).
If inum.eq.-1, then prompt for choice. If inum.lt.-1 output to file
only, not screen.

WINSET(lwin) Set windowing to axis-region on or off (.false. default).

TRUNCF(xmin,xmax,ymin,ymax) Set rectangular truncation (windowing) norm-units.
If all args=0, switch off (default).

PLTEND() Flush graphics, wait for carriage return. Then close, return to text.

\beginsection{Color}

\iftth \special{html:<a href="filltest.f"><img align="right" src="filltest.png"></a>}
COLOR(icol) Set current color of 0-15. Default 15 black (on white).

GRADCOLOR(icol) Set current color from a 0-240 color gradient range.

ACCISGRADINIT(ir1,ig1,ib1,ir2,ig2,ib2) Set the gradient color range,
rgb amplitudes 0-65535.

IGETCOLOR() Integer function.  Get the number of the current
color. [You have to know if it is a 0-15 or a gradient color].

PATHFILL() Fill the path just drawn, with the current color. A path is
a set of vectors all drawn with pen down one after the other.

IDARKBLUE(), IDARKGREEN(),  ISKYBLUE(),  IBRICKRED(),  IPURPLE(),
IGOLD(),  IGRAY(),  ILIGHTGRAY(),  IBLUE(),  IGREEN(),  ICYAN(),
IRED(),  IMAGENTA(),  IYELLOW(),  IBLACK(): Integer functions returning
the 16-color number that gives what their name indicates ($=1-15$).

GRADLEGEND(c1,c2,x1,y1,x2,y2,colwidth,lpara) draw a color gradient
axis-legend. Value corresponding to color goes from c1 to c2: the ends of
the color gradient. Position of axis is (x1,y1) to (x1,y2) in axis-box
units. Width of color bar is colwith times axis-box
width. lpara=.true. says use parallel labels.


\beginsection{Data plotting}

POLYLINE(xv,yv,ilength) Draw a polyline on an already setup graph.

DASHSET(idash) Set dashed line character for polyline. 
(0:solid, 1:long, 2:short, 3:long/short, 4:dots)

LABELINE(xv,yv,ilength,clabel,ilablen) Draw polyline with embedded label
of length ilablen.

YPOLYLINE(yv,ilength) Draw a polyline on an already setup yautoplot.

POLYMARK(xv,yv,ilength,isym) Draw data marked by symbols isym:
1 hollow square, 2 hollow triangle, 3 bullet, 4 solid square, 5 solid 
triangle, 6 hollow star, 7 solid star, 8 hollow diamond, 9 flag, 10 +,
11 x, 12 diamond, 13 -, 14 inverted triangle, 15 circle. Else
char(isym), so ichar('c') gives 'c', ichar('c')+128 gives the font-2
symbol corresponding to 'c' (xi).

POLYDRAW(xv,yv,ilength,DRAWFN) Draw data marked by symbol defined by
the routine DRAWFN, a subroutine with 2 real arguments, being the
normalized x and y position of the symbol center, which draws the symbol.
Specify EXTERNAL DRAWFN in the calling routine. Built-in routines are
ACCIRCLE, ACTRID, ACAST, ACX, ACPLUS.

ACGEN(x,y) is a general routine to be passed to POLYDRAW that will draw a
user-defined symbol without having to define an actual
function. Instead the subroutine ACGSET(xv,yv,ilength,ifill) is called
first with arguments giving the symbol vertices in units of the
charsize, and calling for filling of the symbol if ifill .ne. 0.
Thereafter acgen generates the sybol specified. Since the width and
height are scaled by chrswdth and chrshght, aspect-ratio can be scaled
using charsize().

POLYERR(xv,yv,errv,ilength) Draw error bars yv to yv+errv.

\beginsection{Axes}

AXIS() Automatic x and y axis.

AXIS2() Put tics on the opposite side of the axis box.

XAXIS(first,delta) Draw an axis; labels starting at first, spaced by delta.
If delta=0, autofit labels.

YAXIS(first,delta) If axis is log, first indicates major subtic.

ALTXAXIS(xi,xa) Draw an alternative xaxis whose ends have world values
scaled by the factors xi and xa.

ALTYAXIS(yi,ya) Normally the axis position would be moved by
axptset(1.,1.) and probably ticrev(), all manually, before drawing.

AXLABELS(cxlab,cylab) Label the x/y axes with strings.

BOXTITLE(ctitle) Label the plot with a string centered at the top.

GAXIS(amin,amax, igpow,first,delta, xmin,xmax,ymin,ymax, lpara,laxlog)
General axis with ends positioned at (xmin, ymin) (xmax, ymax) normal
units. World axis labels based on world end values amin, amax, first, delta,
igpow (power of ten label).  Lpara puts labels parallel (else
perp). Laxlog logarithmic.

AXPTSET(xpt,ypt) Set the axis intersection point at specified fractions of
the full lengths of the axes. This determines where the axes are drawn.

TICSET(xlen,ylen,xoff,yoff,ixw,iyw,ixp,iyp) Set tic lengths, label
offsets, widths, decimal points. Labels switched off by width
ixw$\le$0. Defaults set respectively if reals all =0 or ints all
=0. (Defaults: .015,.015,-.03,-.02,4,4,1,1)

TICREV() Reverse the tics and labels to the other side of axis.

TICLABTOG() Toggle the tic labels on and off.

TICNUMSET(itics) Set the number of tics of axis (approximately).

\beginsection{Text plotting}

\iftth \special{html:<img align="right" src="fontshow.png">}

Special text is accessed via ``$\backslash$'' or ``!'' followed by one of 
the following:
{\parindent 5 pc
\item{@}       to normal font 0.
\item{A}       to font 1 (math)
\item{B}       to font 2 (italic)
\item{D}       Toggle subscript mode
\item{U}       Toggle superscript mode
}

JDRWSTR(xn,yn,cstring,just) Draw cstring at (xn,yn)(normal), justified
per parameter just:  0. centered, +1. normal left-justified, etc.

DRCSTR(cstring) Draw cstring from current position. Leave at end of string.

DRWSTR(xn,yn,cstring) Draw cstring from (xn,yn)norm. Leave at end of
string.

WSTR(cstring) Real function: normal-units length of string at current size.

CHARSIZE(width,height) Set character size. (0,0) sets default (.015,.015).

CHARANGL(theta) Set character angle to horizontal in degrees.

GETCANGL(theta) Get character angle to horizontal in degrees.

ANNOTE [DOS only] Enter interactive annotation mode. Type help for help.

GETFONT(fontname) Read a new fontset from file.

LEGENDLINE(xg,yg,nsym,cstring) Draw a legendline at fractional
position in plot box: xg, yg.  Use symbol nsym if positive and $\le$256.
If nsym$<$0 use both line and symbol. If nsym=0 put only line.  
If nsym=257 use nothing but put the string at the usual place.
If nsym=258 use nothing but put the string at the start of the line.

\beginsection{Contouring}

\iftth \special{html:<a href="contest.f"><img align="right" src="contest.png"></a>}

CONTREC(za,cworka,ixm,iym,zclv,icl) Simple contour of entire array za(ixm, iym)
at levels zclv(icl). If icl=0 fit contour levels instead. If icl.lt.0 no
line labels. Cworka is a work character array at least (ixm, iym).

CONTOURL(za,cworka,iLdim,ixm,iym,zclv,icl,x,y,icsw) General contour
za(1:ixm(of ildim), iym) at zclv(icl) (or fit if icl=0, using zclv(1)
if non-zero to determine contour number) on a mesh defined by
x,y. Switch icsw determines call type: 0 equal spacing, x,y not used;
1 vectors x,y determine mesh; 2 arrays x,y determine mesh; bit 5 (16)
sets coloring; bit 6 (32) omits contour lines. Color contouring should
be done with icl .ge. 2 and the maximum and minimum values in zclv(1)
and zclv(icl). Then a gradlegend can be constructed with these values.

CONTOUR(za,xa,ya,ix,iy,zclv,icl) Simple contour over mesh xa,ya of entire
arrays, needing no extra work space. No line labelling.

AUTOCOLCONT(za,iLdim,ixm,iym) Color contour za on rectangular mesh.

\beginsection{Three-dimensional routines}

\iftth \special{html:<a href="hidtest.f"><img align="right" src="hidtest.png"></a>}

HIDWEB(xv,yv,za,iL,ix,iy,isw) Draw a 3-D web of za(ix,iy) (leading
dimension iL) at xv,yv.  View obtained from file eye.dat (3 reals) or
default if nonexistent. Switch isw: Lowest byte .lt.0 no axes;
abs(isw): 0,1 scale to fit; 2 use last scale; 4 don't hide
lines. Second byte (*256) color of web if non-zero. Third byte
(*65536) color of axes (if !=0). (Remains set).

SURF3D(xa,ya,za,iL,ix,iy,isw,worka) Draw a 3-d surface of za(ix,iy),
(leading dimension iL) at xa,ya.  View obtained from file eye.dat (3
reals) or default if nonexistent. Switch isw: Lowest byte .lt.0 no
axes; abs(isw): 0,1 scale to fit; 2 use last scale. Second byte (*256)
color of web if non-zero. Third byte (*65536) color of axes (if
!=0). (Remains set). Array worka(0:iL+1,0:iy+1) must be provided.
Surface is drawn as quadrilaterals filled with color from gradient
according to (za) height scaled to the total height.

AXON(xv,yv,za,iL,ix,iy) Plot axonometrically za(ix,iy) at xv,yv (i.e.
staggered slices at constant y, no perspective).  Use eye.dat data, if
exists; first two values indicate normal distance offset of axis end.

WEBDRW(xv,yv,za,iL,ix,iy,icorner) Draw  a 3-D web using current projection.
Return closest corner to view point in icorner.

SURFDR3(xa,ya,za,iL,ix,iy,worka) Draw 3-D surface using current projection.

HIDINIT(top,bot) Set the top and bottom hiding horizons (usually 0.,1.).

HDPRSET(isw,fxd) Set hiding and 2-3 projection per switch isw. Zero
off.  Positive: hiding only. Negative: hiding and projection with
fixed coordinate either (x,y,or z) for isw=(-1,-2 or -3) having value
fxd. For left handed system, isw=(-4,-5 or -6).

HDPROJECT(ihide,iproject,ifixedcoord,fxdworld,irlsys)
Set hiding (ihide = 1), projecting 2$\rightarrow$3 (iproject.ne.1), which
coordinate to hold fixed (ifixedcoord), what its world value is
(fxdworld), and right (+1) or left (-1) handed system (irlsys).

SCALE3(xmin,xmax,ymin,ymax,zmin,zmax) Set 3-D world scaling to cube size.

SETCUBE(xc,yc,zc,xcen,ycen) Set norm-value of cube edges (+-) and 2-d 
position of cube center.

SCBN(icoord) Real function returns normalized value position of cube face
corresponding to coordinate icoord=constant (icoord =1,2,3). So for
example to project onto the z=constant (negative) face of the cube,
you do: call hdprset(-3,-scbn(3))

TRN32(x,y,z,xt,yt,zt,ifl) If ifl=1, set 3-D transform; (x,y,z) is
point looked at, (xt,yt,zt) is eye. If ifl=2, set axonometric, offset
xt,zt.  If ifl=0 transform (x,y,z) to (xt,yt).

VEC3W(x,y,z,ipen) Draw world vector to (x,y,z).

VEC3N(x,y,z,ipen) Draw norm vector to (x,y,z). Domain (-1.,1.) covers
the cube.

WXYZ2NXYZ(xw,yw,zw,xn,yn,zn) Transform from world 3-D vector to
normalized.

CUBED(icorner) Draw a cube outline when icorner is the nearest.
Code is 1,2,3,4 anticlockwise (from top) from x3min,y3min, + top - bottom.

AXIDENT3() Identify the 3 axes (logically x, y, z).

AX3LABELS(charx,chary,charz) Label the three 3-D axes.

AXPROJ(ic) Draw axes, centered at corner given by ic.
Corner=bits0-3(ic). If bit4(ic)=1, flip labels.  If bit5(ic)=1,
x-labels vertical; if bit6(ic)=1, y-labels vertical, else horizontal.

HIDVECN(x2,y2,ipen) Draw a (2-D) vector, hiding appropriately.

TN2S(px,py,sx,sy) Transform normalized to px,py to screen coordinates
sx,sy. If (world3.h) ihiding is $<0$, project in direction of 3-d axis
mod(abs(ihiding),2) at value fixedn, right/left-handed when
$(1\le-ihiding\le3 )$ or $(4\le-ihiding\le 6)$.

EYE3D(isw) Enter interactive rotation of the view of 3-D drawing. The
routine responds to a mouse drag in the window. On button release the
routine exits with isw=1 if motion occured or isw=0 if no
motion. Normally motion calls for a redraw of the plot and reentering
eye3d, while no motion is the signal to continue. Thus the following
code gives interactive examination of the object:
{\parindent 5em\parskip 0pt
\item{51}Continue
\item{}Drawing commands ...
\item{}call eye3d(isw)
\item{}if(isw.ne.0)goto 51\par
}

IEYE3D() Integer function of convenience to return switch value of eye3d.

\beginsection{Primitives}

VECW(x,y,ipen) Draw world-unit vector to (x,y) with pen up (ipen=0) or down (1).

VECN(x,y,ipen) Draw norm-unit vector to (x,y) with pen up (0) or down (1).

WX2NX(wx), WY2NY(wy) Real functions convert world to normal units.

XN2XW(wx), YN2YW(wy) Real functions convert normal to world units.

GTIC(xgw,ilab,xg,yg,tcos,tsin,axcos,axsin,lpara) Low level tic drawing.

\beginsection{Utilities}

FITRANGE(min,max,itics,ipow,fac10,delta,first,xlast) Fit a maximum of
itics ticks to range (min, max). Decide a scaling factor 10$^{\rm
ipow}$=fac10 and a sensible (scaled) delta between tics. First (xlast)
is integer multiple of delta lying closest to min (max) outside the
range (min,max). So fac10*(first+i*delta) are the tick values.

MINMAX(xv,isize,min,max) Find min and max of vector of size isize.

MINMAX2(xa,iL1,ix,iy,min,max) Find min and max of 2-d array of
leading dimension iL1, over indices (ix,iy).

FWRITE(x,iwdth,ipoint,cstring) Format write the float x into cstring
with specified ipoint decimal places. Total width returned as iwdth.

IWRITE(i,iwdth,cstring) Format the integer i into cstring, return iwdth.

TERMCHAR(cstring) Terminate the string in 0, truncating trailing spaces.

ASLEEP(iusec) Insert a delay of approximately iusec microsec into both
live plotting and ps files, for making sequential drawing animations.
Timing will be much slower over remote Xservers and serious bandwidth
wasted because delays are generated by writing to the screen. When
playing such ps files using ghostscript, the duration can be changed
by an arbitrary (real) factor by predefining the gs macro SF using a
command such as

gs -c /SF 2.5 def -f plot0001.ps

which causes the animation duration to be multiplied by 2.5

\beginsection{Linking}

The routines are normally compiled and gathered into a library which
is subsequently linked against. The interface routines are contained in
vecx.c or vecwin.c. Of these driver files, the routines svga, txtmode,
vec, vecfill, scolor, accisgraddef, acgradcolor, getrgbcolor are
public, for calling by higher level routines, and possibly the user. 

For X window systems, they require the Xlib and Xt libraries, calling
for compiler/linker arguments such as  -L/usr/X11R6/lib/ -lXt -lX11.


\end

